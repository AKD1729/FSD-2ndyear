Carlos Castaneda, The Teaching of Don Juan.
The general notions about human understanding ... which are illustrated by discoveries in atomic physics are not in the nature of thing wholly unfamiliar, wholly unheard of, or new. Even in our own culture they have a history, and in Buddhist and Hindu thought a more considerable and central place. What we shall find is an exemplification, and ecouragement, and refinement of old wisdom.-Julius Robert Oppenheimer.
For a parallel to the lesson of atomic theory... [we must turn] to those kind of epistemoogical problems with which already thinker like the Buddha and Lao Tzu have been confronted, when trying to harmonize our position as spectators and actors in the great drama of existence?-Niels Bohr.
The great scientific contribution in theoretical physics that has come from japan since the last war may be an indication of a certain relationship between philosophical ideas in the tradition of Far East and the philosophical substance of quantum theory.-Werner Heisenberg
When the mind is disturbed, the multiplicity of things is produced, but when the mind is quieted, the multiplicity of thing disappears.-Avidya or Ignorance
He who,dwelling in all things, Yet is other than all things, Whom all things do not know, Whose body all things are, Who controls all things from within-He is your Soul, the Inner Controller, The Immortal.
From the unreal lead me to the real! From darkness lead me to light! From death lead me to immortality!-Brihad-aranyaka Upanishad
Finshing baskets are employed to cath fish: but when the fish are got, the men forget the baskets: snares are employed to cath hares: but when the hares are got, men forget the snares. Words are employed to convey ideas; but when the ideas are grasped, men forget the words.-Taoist Sage Chuang Tzu.
What is soundless, touchless, formless, imperishable, Likewise tasteless, constant, odourless, Without beginning, without end, higher than the great, stable- By discerning That, one is liberated from the mouth of death-Upnishads.
There the eye goes not, Speech goes not, nor the mind. We know not, we understand not How one would teach it-Upnishad.
The Tao that can be expressed is not the eternal Tao.-Lao Tzu(Tao Te Ching).
If it could be talked about, everybody would have told their brother.-Chuang Tzu.
Our normal waking consciousness, rational consciousness as we call it, is but one special type of consciousness, whilst all about it, parted from it by the filmiest of screens, there lie potential forms of consciousness entirely different.-William James.
Rational knowledge and ratonal activities certainly constitute the major part of scientific research woulld, in fact, be useless if it were not complemented by the intution that gives scientists new insights and makes them creative. These insights tend to come suddenly and, characteristically, noth when tend to come suddenly and, characteristically, not when sitting at a desk working out the quations, but when reaxing. In th bath, during a walk in the woods, on the beach, etc. During these periods of relaxation after concetrated intellectual activity, the intutive mind seems to take over and can produce the sudden clarifying insights which give so much joy and delight to scientific research.
The combination of mathematics and theology, which began with pythagoras, characterized religous phislosophy in Greece, in the Middle Ages, and in moderntimes down to Kant ... In Plato, St Augustine, Thomas Aquinas, Descrates, Spinoza and Leibniz there is an intimateblending of religion and reasoning, of moral aspiration with logical admiration of what is timeless, which come from Pythagoras, and distinguishes the intellectualized theology of Europe from the more straightforward mysticism of Asia.
Personal experience is ... the foundation of Buddhist philosophy. In this sense Buddhism is radical empiricism or experientialism, whatever dialectic later developed to probe the meaning of enlightenment-experience.-D.T.Suzuki
He who would understand the meaning of Buddha nature must watch for the season and the causal relations.-Zen verse
The seeing plays the most important role in Buddhist epistemology, for seeing is at the basis of knowing. Knowing is impossible without seeing; all knowledge has its origin in seeing. Knowing and seeing are thus found generally united in Buddha's teaching. Buddhist philosophy therefore ultimately points to seeing reality as it is. Seeing is experiencing enlightenment.-D.T. Suzuki
My predilection is to see ... because only by seeing can a man of knowledge know.-Yaqui mystic Don Juan
Although deep mystical experiences do not, in general, occur without long preparation, direct intutive insights are experienced by all of us in our everyday lives. We are all familar with the situation where we have forgotten the name of a person or place, or some other word, and cannot produce it in spite of the utmost concentration. We have it on the tip of our tongue but it just will not come out, until we give up and shift our attention to something else when suddenly, in a flash, we remember the forgotten name No thinking is involved in the process. It is a sudden, immediate insight. This example of suddenly remembering something is particularly relevant to Buddhism which holds that our original nature is that of the enlightened Buddha and that we have just forgotten it. Students of Zen Buddhism are asked ot discover their 'original face' and the sudden 'remembering' of this face is their enlightenment.
In our daily life, direct intutive insights into the nature of thing are normally limited to extremely brief moments. Not so in Eastern Mysticism where they are extended to long periods and, ultimately, become a constant awareness. The preparation of the mind for this awareness-for the immediate nonconceptual awareness of reality- is the main purpose of meditation.
He who pursues learning will increase every day; He who pursues Tao will decrease every day.-Lao Tzu.
Shikan-taza is a heightened state of concentrated awareness wherein one is neither tense nor hurried, and certainly never slack. It is the mind of somebody facing death. Let us imaging that you are engaged in a duel of swordsmanship of the kind that used to take place in ancient Japan. As you face your opponent you are unceasingly watchful, set, ready. Were you to relax you vigilance even momentarily, you would be cut down instantly. A crowd gathers to see fight. Since you are not blind you see them from the corner of your eye, and since you are not deaf you hear them. but not for an instant is your mind captured by these sense impressions.
As far as the laws of mathematics refer to reality, they are not certain; and as far as they are certain, they do not refer to reality.-aphorism of Einstein.
Myth embodies the nearest approach to absolute truth that can be stated in words.-Ananda Coomaraswamy
When a monk asked Fuketsu Ensho, 'When speech and silence are both inadmissible, how can one pass without error? the master replied: I always remember Kiangsu in March- The cry of the partidge, The mass of fragrant flowers.'
Leaves falling, Lie on one another; The rain beats the rain.
sHindu by cosmic dance of the god Shiva as to the physicist by certain aspects of quantum field theory. Both the dancing god and the physical theory are creations of the mind: models to describe their authors intution of reality.
